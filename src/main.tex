%! Author = nitinsingh
%! Date = 24/12/23

% Preamble
\documentclass[11pt]{article}

% Packages
\usepackage{amsmath,amsthm,amsfonts,amssymb,enumitem,tikz,graphicx}
\usepackage{tikz-cd}
\usetikzlibrary{cd}
%\usepackage{lncs}
\usepackage{adjustbox}

\title{Basic Commutative Algebra: Summary}


%% Sets %%
\newcommand{\Z}{\ensuremath{\mathbb{Z}}}
\newcommand{\N}{\ensuremath{\mathbb{N}}}
\newcommand{\Q}{\ensuremath{\mathbb{Q}}}
\newcommand{\R}{\ensuremath{\mathbb{R}}}
\newcommand{\C}{\ensuremath{\mathbb{C}}}

\newcommand{\ker}[1]{\ensuremath{\mathrm{Ker}\,#1}}
\newcommand{\range}[1]{\ensuremath{\mathcal{R}(#1)}}
\newcommand{\image}[1]{\ensuremath{\mathrm{Im}(#1)}}
\newcommand{\roi}[1]{\ensuremath{\mathcal{O}_{#1}}}
\newcommand{\norm}[2]{\ensuremath{\mathrm{Norm}_{#1/{#2}}}}
\newcommand{\trace}[2]{\ensuremath{\mathrm{Tr}_{#1/{#2}}}}
\newcommand{\det}[1]{\ensuremath{\mathrm{det}(#1)}}


\theoremstyle{plain}
\newtheorem{theorem}{Theorem}[section]
\newtheorem{lemma}{Lemma}[section]
\newtheorem{definition}{Definition}[section]
\newtheorem{example}{Example}[section]


% Document
\begin{document}
\maketitle

The algebraic objects of interest for us in this note are {\em finitely generated abelian groups}. We will
describe the structure of an arbitrary finitely generated abelian group. We then go on to study rings and modules
which are {\em noetherian}, i.e the module and all its submodules are finitely generated. For rings, the equivalent
statement is that the ring and all its ideals are finitely generated.
Informally, finitely generated algebraic structures are interesting as implicitly we always have a finite set of
generators to describe an arbitrary sub-structure (such as a submodule or an ideal). We will highlight the usefulness
of this property as we go along.

\section{Finitely Generated Abelian Group}\label{sec:fin-gen-abe-grp}
Let $(G,+)$ be an abelian group. We say that $G$ is finitely generated if there exist $g_1,\ldots,g_n\in G$ for
some $n\in \N$ such that $G=\{m_1 g_1+\cdots+m_n g_n: m_i\in \Z\}$. Clearly, any finite abelian group is finitely
generated. We also see that a cyclic group is finitely generated (by one generator). Similarly groups like
$(\Z^n,+)$ are finitely generated (by the standard basis). The following theorem is a structure theorem for all
finitely generated abelian groups.

\begin{theorem}[Finitely Generated Abelian Groups]\label{thm:fin-gen-abe-grp}
A finitely generated abelian group $G$ is isomorphic to
\[ (\Z/n_1\Z)\oplus (\Z/n_2\Z)\oplus \cdots \oplus (\Z/n_r\Z)\oplus \Z^s \]
for some $n_1|n_2|\cdots n_r\in \N$ and $s\in \N$.
\end{theorem}
\begin{proof}
Since, $G$ is finitely generated, let us fix the generators $g_1,\ldots,g_n$ of $G$. This allows us to view $G$
as a homomorphic image of $\Z^n$, via the homomorphism $\phi: \Z^n\rightarrow G$ which maps standard basis vector
$\vec{e}_i$ to $g_i$. Then, we have by the isomorphism theorem $\Z^n/\ker{\phi}\cong G$. Clearly, if $\ker{\phi}$
is trivial, we see that we can already conclude that $G\cong \Z^n$. So, to fully understand the structure of $G$,
we must take into account vanishing combinations of generators of $G$.

Let us {\em assume} for
now that $\ker{\phi}$ is also finitely generated. We will prove it later. Let $h_1,\ldots,h_m$ be generators of
$\ker{\phi}$, and let $\psi: \Z^m\rightarrow \ker{\phi}$ be the natural surjective homomorphism from $\Z^m$
to $\ker{\phi}$. Then we have $G\cong \Z^n/\psi(\Z^m)$. Let $A$ denote the $n\times m$ matrix over $\Z$ which
represents the homomorphism $\psi$. Then $G\cong \Z^n/\range{A}$. Now consider the {\em Smith Normal Form} decomposition
of the integer matrix $A$ as $PA'Q$ where $A'$ is $n\times m$  diagonal matrix with
$n_1,\ldots,n_s,0,\ldots,0$ as the diagonal while $P,Q$ are
invertible (over integers) matrices of sizes $n\times n$ and $m\times m$ respectively.Let $\psi'$ denote the
homomorphism $\Z^m\rightarrow \Z^n$given by matrix $A'$, and let $\tau$ and $\sigma$ denote isomorphisms
$\Z^m\rightarrow \Z^m$ and $\Z^n\rightarrow \Z^n$ given by matrices $Q$ and $P$ respectively. We have the
following commutative diagram.

\adjustbox{scale=1.5, center}{%
\begin{tikzcd}[scale=1]
\Z^m \arrow[r, "\psi"] \arrow[d, "\tau"] & \Z^n \arrow[d, "\sigma"] \\
\Z^m \arrow[r, "\psi'"] & \Z^n
\end{tikzcd}
}

Now we have:
\begin{align*}
\Z^n/\psi(\Z^m)\cong \Z^n/\sigma(\psi(Z^m))\cong \Z^n/\psi'(\tau(\Z^m))=\Z^n/\psi'(\Z^m)
\end{align*}
Since, $\psi'$ is represented by the diagonal matrix $A'$, the quotient $\Z^n/\psi'(\Z^m)$ is simple
to describe and is seen to be $(\Z/n_1\Z)\oplus \cdots \oplus (\Z/n_s\Z)\oplus \Z^r$. We now return to
proving the crucial assumption we made about a subgroup of $\Z^n$ (specifically, $\ker{\phi}$); that it
is finitely generated. We prove it seperately as a lemma below.
\end{proof}

\begin{lemma}\label{lem:zn-noetherian}
For $n\in \N$, any subgroup of $(\Z^n,+)$ is finitely generated.
\end{lemma}
\begin{proof}
We prove by induction. For $n=1$, the claim is true as $\Z$ is a PID, and any additive subgroup of $\Z$ is
also an ideal of $\Z$. Consider the exact sequence of homomorphisms:

\adjustbox{scale=1, center}{%
\begin{tikzcd}
0 \arrow[r] & \Z^n \arrow[r, "\psi"] & \Z^n\oplus \Z \arrow[r, "\phi"] \arrow[d, "\pi"] & \Z \arrow[r] & 0 \\
 & & (\Z^n\oplus \Z)/{}_{\ker{\phi}} \arrow[ur, dashleftrightarrow, "\bar{\phi}"] & &
\end{tikzcd}
}
In the above $\psi$ is a natural inclusion homomorphism, while $\phi$ is a natural projection homomorphism. Let
$H$ be a subgroup of $\Z^n\oplus \Z$. Consider $\pi(H)$, the image of subgroup $H$ under the quotient homomorphism
$\pi: a \mapsto a + \ker{\phi}$. We may think of $\pi(H)$ as $H \textrm{ mod } \ker{\phi}$. Since, $\bar{\phi}$
is an isomorphism, we conclude that $\pi(H)$ is isomorphic to subgroup of $\Z$ and hence finitely generated.
Now consider $a\in H$. We have,
$a+\ker{\phi}\in \pi(H)$ and thus $a+\ker{\phi} = m(h+\ker{\phi})= mh+\ker{\phi}$, for some $m\in \Z$ and
$h+\ker{\phi}$ is the generator of $\pi(H)$. Thus $a=mh + \ker{\phi}$, and thus $a-mh\in \ker{\phi}$. We note
that $a-mh\in H$ and thus $a-mh\in H\cap \ker{\phi}$. Since $H\cap \ker{\phi}$ is a subgroup of $\ker{\phi}$,
and thus by exactness of the sequence is isomorphic to subgroup of $\Z^n$. By induction hypothesis,
$H\cap \ker{\phi}$ is finitely generated. This implies $a=mh + m_1 h_1 + \cdots + m_k h_k$ for some $m,m_i\in \Z$
where $h_1,\ldots,h_k$ are generators of $H\cap \ker{\phi}$.
\end{proof}

\section{Noetherian Rings and Modules}\label{sec:noetherian-rings-modules}
Noetherian rings and modules are interesting, as any arbitrary ideal/submodule can be described in terms of
finite number of generators. In this section, we will look at techniques which help us in showing this property,
in particular we will see results which help us in proving complex structures to be noetherian based on the
simpler ones.

\begin{definition}[Noetherian Rings and Modules]\label{defn:noetherian}
An $R$-module $M$ over is called {\em noetherian} if every $R$-submodule of $M$ is finitely generated. A ring
$R$ is called noetherian if it is a noetherian module over itself, i.e, every ideal in $R$ is finitely generated.
\end{definition}

The most basic example of a neotherian ring (and module) is $\Z$. The following is an important technique for
showing modules to be noetherian.

\begin{lemma}\label{lem:ses}
Let
\begin{tikzcd}
0 \arrow[r] & M \arrow[r, "\psi"] & N \arrow[r, "\phi"] & S \arrow[r] & 0
\end{tikzcd}

be a short exact sequence of $R$-modules. Then $N$ is noetherian if and only if both $M$  and $S$ are
noetherian.
\end{lemma}
\begin{proof}
The proof is similar to the proof of Lemma \ref{lem:zn-noetherian}. The essential idea is that a submodule
$H$ of $N$ can be expressed as $H \textrm{ mod } \ker{\phi}$ + $H\cap \ker{\phi}$, with both of those being
finitely generated, being isomorphic to sub-modules of noetherian modules $S$ and $M$ respectively.
\end{proof}

As an application of the above Lemma, let us prove that direct products of noetherian $R$-modules are again
noetherian.

\begin{lemma}\label{lem:direct-product-noetherian}
Let $M$ be a noetherian $R$-module. Then $M\oplus M$ is noetherian. Moreover, $M^n$ is noetherian for all
$n\in \N$.
\end{lemma}
\begin{proof}
Consider the s.e.s:
\begin{tikzcd}[scale=0.5]
0 \arrow[r] & M \arrow[r, hook] & M\oplus M \arrow[r] & M \arrow[r] & 0.
\end{tikzcd}
From Lemma \ref{lem:ses}, $M\oplus M$ is noetherian. By induction, we can extend it to $M^n$ for all $n\in \N$.
\end{proof}
We immediately conclude that $\Z^n$ is a noetherian $\Z$-module for all $n\in \N$, which we earlier showed directly.
Our next lemma makes our life easier. It shows that when considering modules over {\em noetherian} rings, it is
sufficient for module to be finitely generated for it to be noetherian (i.e, we need not show that all submodules
are finitely generated as well).

\begin{lemma}\label{lem:modules-noetherian-ring}
Let $M$ be a finitely generated $R$-module for an noetherian ring $R$. Then $M$ is noetherian.
\end{lemma}
\begin{proof}
Let $M$ be generated by $n$ generators $g_1,\ldots, g_n$. Then there is a surjective homomorphism
$\phi: R^{\oplus n}\rightarrow M$. Consider the s.e.s:

\adjustbox{scale=1,center}{%
\begin{tikzcd}[scale=0.7]
    0 \arrow[r] & \ker{\phi} \arrow[r, hook] & R^{\oplus n} \arrow[r, "\phi"] & M \arrow[r] & 0.
\end{tikzcd}
}

We know that $R^{\oplus n}$ is noetherian $R$-module.
The claim then follows from Lemma \ref{lem:ses}.
\end{proof}

We now show that quotients of noetherian rings and modules are also neotherian.

\begin{lemma}\label{lem:noetherian-quotients}
Let $R$ be a noetherian ring and $I$ be an ideal of $R$.
Then, the quotient $R/I$ is noetherian. Analogous
claim holds for quotient modules.
\end{lemma}
\begin{proof}
Each of $I$, $R$ and $R/I$ are $R$-modules, and we conclude the result from Lemma \ref{lem:ses} using the
sequence:
\begin{tikzcd}[scale=0.7]
0 \arrow[r] & I \arrow[r, hook] & R \arrow[r, "\pi"] & R/I \arrow[r] & 0
\end{tikzcd}
\end{proof}

We now show than an important class of rings, i.e, the polynomial rings over noetherian rings are also
noetherian. This is called the Hilbert Basis Theorem. We need a definition:
\begin{definition}[Finitely Generated Rings over Rings]\label{defn:rings-over-rings}
We say that a ring $S$ is finitely generated as ring over a ring $R\subseteq S$ if there exists a finite subset
$\mathcal{X}$ of $S$ such that each element of $S$ can be written as a $R$-linear combination of
monomials of $\mathcal{X}$.
\end{definition}
We note that if $M\supseteq R$ is a finitely generated $R$-module, then it is also finitely generated as a ring
over $R$.


\begin{theorem}[Hilbert Basis Theorem]\label{thm:hilbert-basis-theorem}
Let $R$ be a noetherian ring. Then the polynomial rings $R[X_1,\ldots,X_n]$ is also noetherian for any $n$.
\end{theorem}
\begin{proof}
It suffices to prove that $R[X]$ is noetherian if $R$ is noetherian. Let $I$ be any ideal of $R[X]$. We need to
exhibit a finite set of polynomials in $I$, such that an arbitrary $f\in I$ can be expressed as their $R[X]$-linear
combination. Our approach will be inductive. First, we identify a set of polynomials $f_1,\ldots,f_m$ in $I$
which will be used to ``annihilate" the leading term of the target polynomial $f$. Then, we can inductively argue
that the reduced polynomial can be expressed as a linear combination of finite set of polynomials. To this end,
we consider the set $A$ of leading coefficients of polynomials in $I$. It can be seen that $A$ must be a ideal
in $R$, and thus is finitely generated ($R$ is noetherian). Let $a_1,\ldots,a_m\in A$ be the generators, and
let $f_1,\ldots,f_m$ be polynomials in $I$ where $a_i$ is the leading coefficient of $f_i$. Let $d$ be the
degree of the polynomial with maximum degree in $f_1,\ldots,f_m$.

As base case, we show that all polynomials in $I$ of degree at most $d$ are finitely generated. Let $S_e$ denote
polynomials in $I$ with degree at most $e$. Note that $S_d$ is an $R$-submodule of the set of all polynomials
in $R[X]$ of degree at most $d$, which is a finitely generated $R$-module, and hence a noetherian module. Thus,
$S_d$ is also finitely generated. Let $h_1,\ldots,h_k$ be the polynomials in $I$ which generate the submodule
$S_d$.

Now, assume that $S_e$ is generated by polynomials $f_1,\ldots,f_m,h_1,\ldots,h_k$ for $e\geq d$. Consider
$f\in S_{e+1}$, and let $a$ be the leading coefficient of $f$. As $A$ is a finitely generated module, we
have $a=r_1 a_1 + \cdots + r_m a_m$ for some $r_i\in R$. It can be seen that the polynomial
$f - \sum_{i=1}^m r_i X^{e+1-deg(f_i)}f_i$ is a polynomial of degree at most $e$ and thus in $S_e$. The claim
now follows from induction.
\end{proof}

The above proof relies on two crucial observations: noetherian property of $R$ is used
to produce a set of polynomials $f_1,\ldots, f_m$ which help us annihilate the leading term of the polynomials
allowing us to induct. Again, we rely on the noetherian property of bounded degree polynomials in $R[X]$ to seed
the induction. We really did not have much more structure on the set $S_d$, aside from the fact it is a submodule
of a noetherian module !!!.


\section{Algebraic Numbers and Algebraic Integers}\label{sec:alg-numbers}
We will call a number $\alpha$ to be {\em algebraic} if it satisfies $p(\alpha)=0$ for some
polynomial $p\in \Q[X]$. Where does this $\alpha$ belong ? We will assume $\alpha$ to come from
an algebraically closed extension field of $\Q$, which we denote by $\overline{\Q}$. We may even
think $\overline{\Q}=\C$.

\begin{definition}[Algebraic Number]\label{defn:alg-number}
We call $\alpha\in \overline{\Q}$ to be an {\em algebraic number} if it is root of a polynomial
$p(X)\in \Q[X]$.
\end{definition}
Examples of algebraic numbers are $\sqrt{2}$, $\sqrt{5}$ and $(1+\sqrt{5})/2$ which satisfy the polynomials
$x^2-2$, $x^2-5$ and $x^2-x-1$ respectively. We now show that product and sums of algebraic numbers are again
algebraic. We first state a useful characterization of algebraic numbers.

\begin{lemma}\label{lem:finite-extension}
A number $\alpha\in \overline{\Q}$ is algebraic if and only if the ring $\Q[\alpha]$ is a
finite dimensional vector space over $\Q$.
\end{lemma}
\begin{proof}
Clearly, when $\alpha$ is an algebraic number $\Q[\alpha]$ is generated by $1,\alpha,\ldots,\alpha^{d-1}$ where
$d$ is the degree of polynomial $p\in \Q[X]$ such that $p(\alpha)=0$.
Conversely, if $\Q[\alpha]$ is a finite dimensional vector space of dimension $d$, the vectors $1,\alpha,\ldots,
\alpha^d$ are linearly dependent over $\Q$, and hence $\alpha$ satisfies a polynomial $p(X)\in \Q[X]$ of degree
$d$.
\end{proof}

Next, we show closure of algebraic numbers under addition and multiplication.
\begin{lemma}\label{lem:alg-num-closed}
Let $\alpha,\beta\in \overline{\Q}$ be algebraic numbers. Then $\alpha+\beta$ and $\alpha\beta$ are also
algebraic numbers.
\end{lemma}
\begin{proof}
Suppose $\alpha,\beta$ satisfy polynomials of degree $m$ and $n$ respectively over $\Q$. Then it is seen that
$\Q[\alpha,\beta]$ is finite dimensional vector space over $\Q$ with dimension at most $mn$ (generated by
monomials of the form $\alpha^i\beta^j$ for $0\i<m$, $0<j<n$). Now $\Q[\alpha+\beta]\subseteq \Q[\alpha,\beta]$,
and therefore $\Q[\alpha+\beta]$ is a finite dimensional vector space over $\Q$, and by Lemma
\ref{lem:finite-extension} it is algebraic. Similar argument holds for $\alpha\beta$.
\end{proof}

Below we illustrate a computational insight into finding a polynomial satisfied by an algebraic number such
as $\sqrt{2}+\sqrt{5}$. Let $\alpha=\sqrt{2}+\sqrt{5}$. We compute successive powers of $\alpha$, and note that
the surds in the expansions are from the set $\{\sqrt{2},\sqrt{5}, \sqrt{2}\sqrt{5}\}$. After sufficient powers,
we can simply ``eliminate'' the square-roots and get a relation amongst powers of $\alpha$ using rational coefficients.

\begin{align*}
\alpha &= \sqrt{2} + \sqrt{5} \\
\alpha^2 &= 7 + 2\sqrt{2}\sqrt{5} \\
\alpha^3 &= 17\sqrt{2} + 11\sqrt{5} \\
\alpha^4 &= 89 + 28\sqrt{2}\sqrt{5}
\end{align*}

Eliminating $\sqrt{2}\sqrt{5}$ from second and fourth equations we have $(\alpha^4 - 89)/28=(\alpha^2-7)/2$ which
gives a rational polynomial satisfied by $\alpha$.

We now define algebraic integers, which are counterparts of $\Z$ in $\Q$.
\begin{definition}[Algebraic Integer]\label{defn:alg-integer}
We say that $\alpha\in \overline{\Q}$ is an algebraic integer if it is a root of a {\em monic} polynomial
$p\in \Z[X]$.
\end{definition}
We see that $\sqrt{2}$ is an algebraic integer as it is the root of integral polynomial $x^2 - 2$. We now
give analogous characterization for algebraic integers.

\begin{lemma}\label{lem:alg-integers-module}
An algebraic number $\alpha\in \overline{\Q}$ is an algebraic integer if and only if the ring $\Z[\alpha]$
generated by it is a finitely generated $\Z$ module.
\end{lemma}
\begin{proof}
Clearly, if $\alpha$ is a root of a monic integer polynomial $f$, $\Z[\alpha]$ is generated by $1,\alpha,\ldots,\alpha^{d-1}$,
where $d=deg(f)$. Conversely, suppose $\Z[\alpha]$ is finitely generated, and let $f_1(\alpha),\ldots,f_n(\alpha)$
be the generators. Choose $d$ larger than the degrees of polynomials $f_1,\ldots,f_n$. Then we can write
$\alpha^d = \sum_{i=1}^n m_if_i(\alpha)$. Thus $\alpha$ is a root of monic integer polynomial $x^d-\sum_{i=1}^n m_i f_i$.
\end{proof}

We now show that algebraic integers are closed under addition and multiplication.

\begin{lemma}\label{lem:alg-integer-closure}
Let $\alpha,\beta\in K$ be algebraic integers in a number field $K$. Then $\alpha+\beta$ and $\alpha\beta$
are also algebraic integers in $K$.
\end{lemma}
\begin{proof}
Let $\alpha$ and $\beta$ be roots of integer polynomials of degrees $m$ and $n$ respectively. Then, we see
that the $\Z$-module $\Z[\alpha,\beta]$ is finitely generated by monomials $\alpha^i\beta^j$ with $0\leq i<m$,
$0\leq j<n$. Since $\alpha+\beta\in \Z[\alpha,\beta]$, $\Z[\alpha+\beta]$ is a $\Z$-submodule of
$\Z[\alpha,\beta]$. Since $\Z[\alpha,\beta]$ is noetherian, the submodule $\Z[\alpha+\beta]$ is finitely generated.
Thus $\alpha+\beta$ is an algebraic integer. Similar argument shows $\alpha\beta$ to be an algebraic integer.
\end{proof}

The computational approach for finding an integer polynomial for $\sqrt{2}+\sqrt{5}$ no longer works. This is
because we can no longer ``eliminate" the square-roots over $\Z$. Thus, the Lemma \ref{lem:alg-integer-closure}
essentially gives an existential characterization, without implying an algorithmic scheme.

\section{Ring of Integers}\label{sec:ring-of-integers}
We say that $K$ is a number field if it is a finite degree extension field of $\Q$, i.e $[K:\Q] < \infty$.
Let $K$ be a number field. By {\em primitive element theorem}, there exists $\alpha\in K$ such that
$K=\Q(\alpha)$. Moreover, there exists an irreducible polynomial $f\in\Q[X]$ of degree $n=[K:\Q]$ such
that $f(\alpha)=0$. We now define the ring of integers:

\begin{definition}[Ring of Integers]\label{defn:ring-of-integers}
Let $K$ be a number field. The ring of integers of $K$, denoted by $\roi{K}$ is defined as:
\begin{equation*}
\roi{K} = \{\alpha \in K: \alpha \text{ is an algebraic integer }\}
\end{equation*}
\end{definition}
As an example, it can be seen that for $K=\Q$, $\roi{K}=\Z$. One can also show that for
$K=\Q(i)$, where $i$ is a root of the polynomial $X^2+1$, we have $\roi{K}=\Z[i]$. In general,
however it is not true that $\Z[\alpha]$ is the ring of integers of $\Q(\alpha)$. In fact,
we can easily show that $\Z[\sqrt{5}]$ is not the ring of integers for $Q(\sqrt{5})$ (the
ring of integers in this case is $\Z[\varphi]$ where $\varphi=(1+\sqrt{5})/2$). Minimal
polynomials are often useful in computing the ring of algebraic integers. We state a lemma
and illustrate its usefulness with an example.

\begin{lemma}\label{lem:min-poly}
A number $a\in K$ is an algebraic integer if and only if its minimal polynomial has coefficients in $\Z$.
\end{lemma}

\begin{example}\label{ex:q-root-5}
The ring of integers of the number field $\Q(\sqrt{5})$ is given by $\Z[\varphi]$ for $\varphi=(1+\sqrt{5})/2$.
\end{example}
\begin{proof}
Let $x=a+b\sqrt{5}\in \Q(\sqrt{5})$ be an algebraic integer. To determine the minimal polynomial of $x$, we
consider the linear map $T_x: Q(\sqrt{5})\rightarrow Q(\sqrt{5})$ which maps $t$ to $x\cdot t$. The matrix
of $T_x$ with respect to the basis $\{1, \sqrt{5}\}$ of $\Q(\sqrt{5})$ is given by:
\begin{align*}
A &= \begin{pmatrix}
        a & 5b \\
        b & a
    \end{pmatrix}
\end{align*}
Using Cayley-Hamilton theorem, the minimial polynomial of $T_x$ (and thus also of $x$) is given by
$(X-a)^2 - 5b^2= X^2 -2aX + (a^2 - 5b^2)$. We require $2a$ and $a^2 - 5b^2$ to be integers. Thus
$a=k/2$ for some $k\in \Z$ and $k^2/4 - 5b^2\in \Z$, and thus $5b^2\in \frac{1}{4}\Z$. This implies whenever
$a$ has denominator as $2$, $b$ also has denominator as $2$. This proves that ring of algebraic integers
is generated by $(1+\sqrt{5})/2$.
\end{proof}

In general though, the ring of integers of a number field is difficult to characterize. We will, therefore
introduce a somewhat simpler ring structure called an {\em order}.

\begin{definition}[Order]\label{defn:order}
An order in $\roi{K}$ for a number field $K$ is a subring $R$ of $\roi{K}$ such that the quotient $\roi{K}/R$
is finite.
\end{definition}
We will show later that for $K=\Q(\alpha)$, the ring $\Z[\alpha]$ is an order in $\roi{K}$. Now, we show
how $\roi{K}$ resembles the ring $\Z$, in that some integer multiple of an element in $K$ eventually lands
up in $\roi{K}$ !!!.

\begin{lemma}\label{lem:roi-q-intersection}
Let $K$ be a number field with $\roi{K}$ as its ring of integers. Then $\roi{K}\cap \Q=\Z$ and
$\Q \roi{K}=K$.
\end{lemma}
\begin{proof}
It is not difficult to see that for $a/b\in \Q$ with $\mathrm{gcd}(a,b)=1$ and $b\neq 1$, $\Z[a/b]$ is not
finitely generated. Thus $\roi{K}\cap \Q=\Z$. Next, we show that for $a\in K$, there exists integer $d\in \Z$
such that $da\in \roi{K}$. We use the minimal polynomial $f(X)$ of $a$. For an integer $d$, we note that
$d^{\mathrm{deg}(f)}f(X/d)$ is the minimal polynomial of $da$. We choose $d$ to be the least common multiple of
all the denominators of the coefficients of $f$, which ensures that $d^{\mathrm{deg}(f)}f(X/d)$ is a monic polynomial
in $\Z[X]$. Thus $da$ is integral, i.e. $da\in \roi{K}$. This implies $\Q \roi{K}=K$.
\end{proof}

Next, we will introduce two linear functionals, {\em norm} and {\em trace} respectively, which are important in
the study of algebraic integers. As a heads up, both the linear functionals take values in $\Z$ for algebraic
integers. We make a definition for an arbitrary pair of number fields.

\begin{definition}[Norm and Trace]\label{defn:norm-and-trace}
Let $K\subseteq L$ be number fields with $[L:K]=d$. Define a $K$-linear map $\ell_a: L\rightarrow L$ (viewing
$L$ as a $d$-dimensional vector space over $K$) by $x\mapsto a\cdot x$. Then we define $\norm{L}{K}(a)=\det{{\ell_a}}$ and
$\trace{L}{K}(a)=\mathrm{Tr}(\ell_a)$. It can be seen that both the operators map $L$ into the field $K$.
\end{definition}

We can describe the norms and traces in terms of orbit of the element under the Galois group of embeddings.

\begin{lemma}\label{lem:norm-and-trace}
Let $K\subseteq L$ be number fields with $[L:K]=d$. Let $\sigma_1,\ldots,\sigma_d$ denote distinct $K$-embeddings
of $L$ into $\overline{\Q}$. Then we have:
\begin{align*}
\norm{L}{K}(a) = \prod_{i=1}^d \sigma_i(a), \quad \trace{L}{K}(a) &= \sum_{i=1}^d \sigma_i(a).
\end{align*}
\end{lemma}
Before, we proceed with the proof, we notice that if $a$ is an algebraic integer, then $\norm{L}{K}(a)$
and $\trace{L}{K}(a)$ are also elements of the integer ring $\roi{K}$. In particular when
$K=\Q$, we have $\trace{L}{\Q}(a)\in \Z$.
Now, we proceed with the proof.
\begin{proof}
Let $f(X)\in K[X]$ be the minimal polynomial of $a\in L$. Let $n$ be the degree of $f$, and
thus $[K(a): K]=n$. We note that $f(\ell_a)\equiv 0$ for linear operator $\ell_a: L\rightarrow L$
defined by $\ell_a(x)=a\cdot x$. If we consider the restriction of operator $\ell_a$ to $K(a)$, which
we denote by $\tilde{\ell}_a: K(a)\rightarrow K(a)$, we can claim  $f(X)$ to be its characteristic
polynomial (characteristic polynomial of $\tilde{\ell}_a$ is of degree $n$, and is divisible
by $f$, hence is same as $f$). Let $n'=d/n$. First, we will claim that the characteristic polynomial
of $\ell_a$ is the polynomial $f^{n'}$. Clearly, the set $\{1,a,\ldots,a^{n-1}\}$ is a basis of $K(a)$
over $K$. Let $b_0,\ldots,b_{n'-1}$ be a basis of $L$ over $K(a)$. Then $\{a^ib_j\}$ with $0\leq i\leq n-1$
and $0\leq j\leq n'-1$ is a basis of $L$ over $K$. Let $V_j$, $0\leq j\leq n'-1$
denote the vector space spanned by basis vectors $\{a^ib_j\}_{i=0}^{n-1}$. We note that $\ell_a$ maps $V_j$ to
$V_j$ for all $j$, and moreover its action is identical on all $V_j$. More specifically, the matrix for the
restriction $\ell_a: V_j\rightarrow V_j$ with respect to ordered basis $\{a^ib_j\}_{i=0}^{n-1}$ is identical
for all $j$. This proves that $f(X)^{n'}$ is the characteristic polynomial of $\ell_a$. It follows that
\begin{equation}\label{eq:det-la}
\det{\ell_a} = \left(\prod_{i=1}^n \gamma_i(a)\right)^{n'}
\end{equation}
where $\gamma_1,\ldots,\gamma_n$ are the $K$-embeddings of $K(a)$ into $\overline{\Q}$. Now let $\sigma_1,\ldots,
\sigma_d$ be the $K$-embeddings of $L$ into $\overline{\Q}$. We note the following: $\sigma_i(a)\in$
$\{\gamma_1(a),\ldots,\gamma_n(a)\}$ for all $i\in [d]$ and (ii) for all $j\in [n]$, $\sigma_i(a)=\gamma_j(a)$
for exactly $n'$
values of $i$. Together (i) and (ii) imply that $\det{\ell_a}=\prod_{i=1}^d \sigma_i(a)$. We now prove both the
observations. Note that (i) follows from the fact that a $K$-embedding of $L$ restricted to $K(a)$ is a $K$-embedding
of $K(a)$ into $\overline{\Q}$. Now for (ii) observe that a $K$-embedding $\gamma$ of $K'=K(a)$ can be extended to
embedding of $L=K'(x)$ in exactly $n'$ ways corresponding to extension $x\mapsto \tau(x)$ where $\tau$ is a $K'$-embedding
of $L$ into $\overline{\Q}$. We cover this in detail in the detour on field embeddings below.

%More precisely, each $\sigma\in \{\sigma_1,\ldots,\sigma_d\}$ is of the form $\tau\circ \gamma$
%where $\gamma$ is $K$-embedding of $K'$ into $\overline{\Q}$, $K''=\gamma(K')$, and $\tau$ is a $K''$-embedding of
%$K''(x)$ into $\overline{\Q}$, i.e., we have:
%$\gamma: K'(x) \leftrightarrow K''(x)$ given by $\gamma:\sum_{i=1}^m a_ix^i \mapsto \sum_{i=1}^m \gamma(a_i)x^i$
%and $\tau$ is $K''$-embedding of $K''(x)$ into $\overline{\Q}$.
\end{proof}

\subsection{Detour on Field Embeddings}\label{subsec:embeddings}
For fields $K$ and $L$, an {\em embedding} is a ring homomorphism $\sigam: K\rightarrow L$. It also follows that an
embedding is {\em injective}. Let $f(X)=f_0+f_1 X+\cdots+f_k X^k\in K[X]$ be a polynomial. We define $\sigma f$ to be the polynomial
$\sigma(f_0) + \sigma(f_1)X+\cdots + \sigma(f_k)X^k$. It must be noted that $\sigma f\in (\sigma F)[X]$.

Let $\alpha$ be algebraic over $K$ and let $\sigma: K(\alpha)\rightarrow L$ be a field embedding.The
following observations are immediate:
\begin{itemize}[leftmargin=2em]
\item If $f\in K[X]$ is a polynomial, then $\sigma(f(\alpha))=(\sigma f)(\sigma \alpha)$.
\item If $\alpha$ is a root of $f\in K[X]$, then $\sigma(\alpha)$ is a root of $\sigma f$.
\item The embedding $\sigma$ is completely specified by $\sigma_{\restriction K}$ and $\sigma(\alpha)$.
\end{itemize}

We now discuss how embeddings of a field can be {\em extended} to embeddings of field extensions.

\begin{lemma}\label{lem:extend-embedding}
Let $\sigma:K\rightarrow L$ be a field embedding. Let $p\in K[X]$ be an irreducible polynomial and $\alpha$ be a root of $p$ in some
extension field of $K$. Then for each $\beta\in L$ such that $\beta$ is a root of $\sigma p$, there exists an embedding $\tau:K(\alpha)\rightarrow L$,
which {\em extends} $\sigma$ and satisfies $\tau \alpha = \beta$. Moreover, for any embedding $\tau:K(\alpha)\rightarrow \bar{A}$ for
algebraic closure $\bar{A}$ of $K$, which extends $\sigma$, $\tau \alpha$ is a root of $\sigma p$.
\end{lemma}
\begin{proof}
First, assume that $\beta\in L$ is a root of $\sigma p$. Define $\tau: K(\alpha)\rightarrow L$ by defining $\tau(f(\alpha))=(\sigma f)(\beta)$,
where $f\in K[X]$ is a polynomial. We show that $\tau$ is well-defined, i.e, if $f(\alpha)=g(\alpha)$, then $(\sigma f)(\beta)=(\sigma g)(\beta)$.
Now, $f(\alpha)=g(\alpha)$ implies $f-g=ph$ for some polynomial $h$. Thus, $(\sigma f - \sigma g)(\beta)=(\sigma p)(\beta)\cdot (\sigma h)(\beta)=0$.
Now, it is trivially verified that $\tau(f(\alpha) g(\alpha))=\tau(f(\alpha))\tau(g(\alpha))$ and similarly for addition. Thus $\tau$ is an embedding.
Next, let $\tau: K(\alpha)\rightarrow \bar{A}$ be an extension of $\sigma: K\rightarrow L$. Then, $\tau \alpha$ is a root of $\tau p=\sigma p$. This
completes the proof of the lemma.
\end{proof}

Our next lemma, shows that we can indeed {\em count} the number of extensions of an embedding, to those over extension fields.
\begin{lemma}\label{lem:extend-embedding-extension}
Let $\sigma:K\rightarrow \bar{A}$ be an embedding of $K$ into an algebraically closed field $\bar{A}$ of characteristic $0$. Let $L$ be a finite extension of $K$ with
$[L:K]=n$. Then, there are exactly $n$ extensions of $\sigma$ to an embedding $\tau:L\rightarrow \bar{A}$.
\end{lemma}
\begin{proof}
Let $L=K(\alpha_1,\ldots,\alpha_r)$. Let $K_1=K(\alpha_1)$, and let $p\in K[X]$ be the irreducible polynomial with $\alpha_1$ as the root (this is
the generator of the ideal containing all polynomials in $K[X]$ which vanish at $\alpha_1$), and let $d_1$ denote its degree. First we note that
$p$ cannot have repeated roots. This is because the field $\bar{A}$ has characteristic $0$ and therefore $p'\neq 0$ will also vanish at $\alpha_1$
contradicting the minimality of $p$. Now, by Lemma ~\ref{lem:extend-embedding} there are exactly $d_1$ extensions $\tau_1:K(\alpha_1)\rightarrow \bar{A}$
of $\sigma$. Now, $L=K_1(\alpha_2,\ldots,\alpha_r)$ and we use induction hypothesis to conclude that there are exactly $[L:K_1]$ extensions of an embedding
$\tau_1$ to $\tau:L\rightarrow \bar{A}$. This proves that there are $d_1[L:K_1]=[L:K]$ extensions of $\sigma$.
\end{proof}

\noindent{\em Remark}: We note that we crucially exploited that minimal polynomial of $\alpha_1$, i.e. $p$ has no repeated roots in $\bar{A}$ (by definition
of algebraically closed field, it splits completely in $\bar{A}$.) Instead of the restriction, we can directly formulate the property of field extension
$L$ of $K$ that minimal polynomials of all $\alpha\in L$ have distinct roots in some algebraic closure of $K$. We call such a field $L$ as
{\em seperable} finite extension of $K$. Thus, the above theorem is also true for {\em seperable} finite extensions.


As a nice application of field embeddings, we show that all finite extensions over fields of characteristic $0$ are in fact simple extensions. It will
suffice to prove the simple version below (the more general claim follows by induction).

\begin{lemma}\label{lem:simple-extension}
Let $L=K(\alpha,\beta)$ be a seperable extension of $K$ with $K$ having characteristic $0$. Then there exists $\gamma\in L$ such that $L=K(\gamma)$.
\end{lemma}
\begin{proof}
Let $n=[L:K]$ and let $\sigma_1,\ldots,\sigma_n$ be the embeddings of $L$ into some algebraic closure of $K$. Note that we are through
if we can exhibit a $\gamma\in L$ such that its orbit under the embeddings, i.e. $\sigma_1(\gamma),\ldots,\sigma_n(\gamma)$, consists of
distinct elements. Then, by Lemma ~\ref{lem:extend-embedding-extension}, $[K(\gamma):K]\geq n$, which together with $K(\gamma)\subseteq L$
implies $L=K(\gamma)$. We consider the polynomial:
\begin{equation*}
p(X) = \prod_{i=1}^n \prod_{j\neq i}\big(\sigma_j\alpha -\sigma_i \alpha + X(\sigma_j \beta -\sigma_i\beta)\big)
\end{equation*}
Clearly, $p(X)$ is a non-zero polynomial, and therefore there exists $c$ such that $p(c)\neq 0$. This is where we use that $K$ is a characteristic
$0$ field, and thus non-zero polynomial $p$ cannot vanish on entire $L$. For such a $c$, we have $\sigma_i(\alpha + c\beta)\neq \sigma_j(\alpha+c\beta)$
for $i\neq j$. Thus $\gamma=\alpha+c\beta$ has distinct images under the embeddings, and thus $L=K(\gamma)$.
\end{proof}

\subsubsection{Normal Extensions}\label{subsubsec:normal-extensions}
A particularly useful finite extension of a field $K$ is one, such that all of its $K$-embeddings into the algebraic closure turn out
to be {\em automorphisms}, i.e., the embeddings never leave the extension.
\begin{definition}[Normal Extension]\label{defn:normal-extension}
A finite extension $L$ of $K$ is said to be {\em normal} if every $K$-embedding of $L$ into some algebraic closure of $K$ is an automorphism
of $L$.
\end{definition}
The definition clearly implies that for all $a\in L$, $L$ also contain all of its conjugates, i.e $\sigma(a)\in L$ for all $K$-embeddings $\sigma$
of $L$. A nice characterization of a normal extension is in terms of {\em splitting} fields of polynomials.

\begin{definition}[Splitting Field]\label{defn:splitting-field}
Let $f\in K[X]$ be a polynomial. By splitting field of $f$, we mean a finite extension $L$ of $K$ such that $f$ has factorization into linear factors
in $L$
\end{definition}

We have the following Lemma relating normal extensions and splitting fields.
\begin{lemma}\label{lem:normal-extn-splitting-fields}
A finite extension $L$ of a field $K$ is a {\em normal} extension if and only if $L$ is the splitting field of some polynomial $p\in K[X]$.
\end{lemma}.

A {\em normal closure} of a field $K$, is the smallest finite extension of $K$ which is also a normal extension. We have the following:
\begin{lemma}\label{lem:normal-closure}
Let $L$ be a finite extension of $K$. Then the normal closure of $L$ is also a finite extension of $K$.
\end{lemma}

This concludes our detour into the field embeddings. We return to investigating the norms and traces over field extensions.

\subsection{Tower Property for Norms and Traces}\label{subsec:norms-tower-prop}
We show that norms and traces satisfy the tower property with respect to field extensions:
\begin{lemma}\label{lem:norm-tower-prop}
Let $K\subseteq L\subseteq M$ be finite extensions with algebraic closure $\overline{\Q}$. Then for all $a\in M$, we have:
\begin{equation*}
\norm{M}{K}(a) = \norm{L}{K}\big(\norm{M}{L}(a)\big),\quad \trace{M}{K}(a) = \trace{L}{K}\big(\trace{M}{L}(a)\big)
\end{equation*}
\end{lemma}
\begin{proof}
Let $m=[M:L]$, $n=[L:K]$ and $d=[M:K]$ where $d=mn$. Let $\gamma_1,\ldots,\gamma_m$ be embeddings of $M$ which fix $L$,
$\tau_1,\ldots,\tau_n$ be embeddings of $L$ which fix $K$, and let $\sigma_1,\ldots,\sigma_d$ be the embeddings of $M$
which fix $K$. Let $\overline{M}$ be the {\em normal closure} of $M$, i.e it is the smallest normal extension of $M$. We
note that $\overline{M}$ is a finite extension of $M$, and hence also of $L$. For $i\in [n]$, define $\tilde{\tau}_i$ to
be the extension of the embedding $\tau_i:L\rightarrow \overline{Q}$ to embedding $\tilde{\tau}_i:\overline{M}\rightarrow \overline{\Q}$.
Note that $\tilde{\tau}_i$ fixes $K$ for all $i\in [n]$. Note that for $i\in [n]$ and $j\in [m]$, $\tilde{\tau}_i\circ \gamma_j$ is an
embedding of $M$ into $\overline{\Q}$ which fixes $K$. Thus $\{\sigma_1,\ldots,\sigma_d\}=\{\tilde{\tau}_i\circ \gamma_j: i\in [n],j\in [m]\}$.
Now, using Lemma ~\ref{lem:norm-and-trace}, we have:
\begin{align*}
\norm{M}{K}(a) &= \prod_{i=1}^d \sigma_i(a) \\
    &= \prod_{i=1}^n \prod_{j=1}^m \tilde{\tau}_i(\gamma_j(a)) \\
    &= \prod_{i=1}^n \tilde{\tau}_i\big(\prod_{j=1}^n \gamma_j(a)\big) \\
    &=  \prod_{i=1}^n \tau_i(\norm{M}{L}(a)) \\
    &= \norm{L}{K}\big(\norm{M}{L}(a)\big)
\end{align*}
\end{proof}





\end{document}